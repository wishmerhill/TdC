\documentclass[a4paper,11pt]{book}
\usepackage[toc,page]{appendix}
\usepackage[T1]{fontenc}
\usepackage[latin1]{inputenc}
\usepackage[italian]{babel}
\usepackage{amsmath}
\usepackage{amsfonts}
\usepackage[big]{layaureo}
\usepackage{indentfirst}

\begin{document}

\chapter{La sicurezza strutturale}
Una struttura non � mai sicura al 100\%, ma ha una sua possibilit� di collasso. L'analisi che si fa � di tipo probabilistico e tende a minimizzare o a rendere accettabile la probabilit� di collasso della struttura stessa.

\paragraph{Metodi di misura della probabilit� di rovina} \hfill \\

Ordinando per complessit� calante si ha:

\begin{description}
\item[liv. 3 - 4 -] metodi esatti;
\item[liv. 2 -] metodo $\beta$;
\item[liv. 1 -] metodo semi-probabilistico agli stati limite;
\item[liv. 0 -] metodo alle tensioni ammissibili $\longrightarrow$ calcolo a rottura.
\end{description}

Al calare della complessit�, diminuisce la precisione e quindi si introducono parametri via via pi� cautelativi. Per i nostri scopo, il liv. � sufficiente. La normativa italiana prevede solo i livelli 1 e 0, mentre l'Eurocodice solo 1 e 2, accennando appena agli altri.

\paragraph{Probabilit� di rovina}

Cos'� la probabilit� di rovina? \'{E} statistica, statistica di una funzione a moltissime variabili. Per semplificare, consideriamo una funzione a due variabili: la resistenza in tutte le sue forme, che indichiamo con $r$, e lo stato di sollecitazione $s$.

\begin{equation}
f_{x,y}=f(s,r)
\end{equation}

In tutti casi in cui $r>s$ la struttura, in generale, star� bene. Viceversa, se $r<s$ avr� il collasso della struttura.

La probabilit� di rovina � il volume sotteso dalla curva di densit� di probabilit� nel dominio non sicuro.

\paragraph{Coefficiente di sicurezza:} \begin{equation}
F_s = \frac{R}{S} \text{   } \left( \dfrac{\text{resistenza}}{\text{sollecitazione}} \right)
\end{equation}

Per $ F_s <1 $ ho collasso. La probabilit� di rovina � la probabilit� che sia $F_s <1$:

\begin{equation}
P_f = P \lbrace R/S \leq 1 \rbrace = P \lbrace F_s \leq 1 \rbrace
\end{equation}

\paragraph{Margine di sicurezza} \begin{equation}
M_s = R - S
\end{equation}

Per $M_s < 0$ ho collasso. La probabilit� di collasso � la probabilit� che sia $M_s < 0$.

\begin{equation}
P_f = P \lbrace R -S \leq 1 \rbrace = P \lbrace M_s \leq 1 \rbrace
\end{equation}

Il metodo di livello 2 lavora proprio su $M_s$, ne calcola i momenti del 1� e del 2� ordine (media $\eta_M$ e scarto $\sigma_M$) e fa il rapporto $\beta = \eta_M / \sigma_M$ e dice de deve valere:

\begin{equation}
\beta \geq \beta^*,
\end{equation}

dove $\beta^*$ � fornito dall'Eurocodice 0.

Se $R$ e $S$ sono distribuzioni normali e indipendenti, vale che
\begin{equation}
\beta = \frac{\eta_R - \eta_S}{\sqrt{\sigma_r^2 + \sigma_s^2}}
\end{equation}

$R$ e $S$ sono pertanto, per il metodo 2, statisticamente indipendenti. In realt� ci� non � quasi mai vero: ad esempio, il peso proprio  del CLS incide sia sulla resistenza che come sollecitazione.

*** completare ***

\section{Livello 1 - stati limite}

\chapter{Elemento inflesso}

\section{Calcolo elastico dell'elemento semplicemente inflesso}
Ipotesi:
La sezione ruota e trasla e si mantiene piana.

La sezione resistente � costituita dal calcestruzzo compresso (al di sopra dell'asse neutro quindi) e dalle armature (compresse e tese). Trascuro la resistenza a trazione del calcestruzzo. 

* non vale la sovrapposizione degli effetti * [NdA: perch�?]

Faccio la verifica sulla massima compressione per calcestruzzo e acciaio, per la massima trazione solo per l'acciaio.


Essendo in campo elastico, le relazioni tra $\varepsilon$ (deformazioni) e $\sigma$ (tensioni) sono lineari: significa che, trovato un valore della tensione o della deformazione, li ho trovati tutti.
\`{E} solo questione di trovare:
\begin{itemize}
  \item asse neutro (si tratta di una retta in un piano, quindi in generale due incognite);
  \item tensione in un punto (una incognita),
\end{itemize}

per un totale di tre incognite.
Le equazioni offerte dal sistema sono 3:
\begin{itemize}
 \item equilibrio alla traslazione;
 \item equilibrio alla rotazione attorno all'asse di sollecitazione \textit{s-s};
 \item equilibrio alla rotazione attorno all'asse neutro \textit{n}.
\end{itemize}

Equilibrio alla traslazione:

\begin{equation}
 \int_{A_c} \sigma d A_c + \sum_{i=1}^m \sigma_{s_i} A_{s_i} = 0 ,
\end{equation}

equilibrio alla rotazione attorno all'asse di sollecitazione:
\begin{equation}
 \int_{A_c} \sigma x d A_c + \sum_{i=1}^m \sigma_{s_i} x_{s_i} A_{s_i} = 0
\end{equation}

ricordando che $x$ \`{e} la distanza dell'elemento dall'asse di sollecitazione $s-s$, e infine l'equilibrio alla rotazione attorno all'asse neutro $n-n$:

\begin{equation}
 \int_{A_c} \sigma \lambda d A_c + \sum_{i=1}^m \sigma_{s_i} \lambda{s_i} A_{s_i} = M \cos \alpha.
\end{equation}

A questo punto posso esprimere le distanze dall'asse neutro in funzione della $ \lambda_c $, ovvero la distanza della fibra superiore pi distante. Per la linearita, ho:

\begin{align*}
 \frac{\sigma_c}{\lambda_c} & = \frac{\sigma}{\lambda}  = \frac{\sigma_{s_i}}{n \lambda_{s_i}} \\
 \sigma & = \frac{\sigma_c}{\lambda_c} \lambda \\
 \sigma_{s_i} & = n \frac{\sigma_c}{\lambda_c}  {\lambda_{s_i}}
\end{align*}

Le equazioni di cui sopra, espresse in funzione di $\sigma_c$ e $\lambda_c$, diventano:

\begin{align}
 & \int_{A_c} \frac{\sigma_c}{\lambda_c} \lambda d A_c + \sum_{i=1}^m n \frac{\sigma_c}{\lambda_c}  {\lambda_{s_i}} A_{s_i}  = 0 , \nonumber \\
 & \int_{A_c} \frac{\sigma_c}{\lambda_c} \lambda x d A_c + \sum_{i=1}^m n \frac{\sigma_c}{\lambda_c}  {\lambda_{s_i}} x_{s_i} A_{s_i}  = 0  \label{eqn:eq_yc} \\
 & \int_{A_c} \frac{\sigma_c}{\lambda_c} \lambda^2 d A_c + \sum_{i=1}^m n \frac{\sigma_c}{\lambda_c}  {\lambda_{s_i}}^2  A_{s_i}   = M \cos \alpha. \nonumber 
 \end{align}

Opero un cambio di variabile, esprimendo le $\lambda$ in funzione di y:

\begin{align*}
 \lambda & = y \cos \alpha \\
 \lambda_{s_i} & = y_si \cos \alpha \\
 \lambda_c & = y_c \cos \alpha
\end{align*}

Le \ref{eqn:eq_yc} possono essere riscritte come:

\begin{align}
 & \int_{A_c} \frac{\sigma_c}{y_c \cos \alpha} y \cos \alpha d A_c + \sum_{i=1}^m n \frac{\sigma_c}{y_c \cos \alpha}  {y_{s_i} \cos \alpha} A_{s_i}  = 0 , \nonumber \\
 & \int_{A_c} \frac{\sigma_c}{y_c \cos \alpha} y \cos \alpha x d A_c + \sum_{i=1}^m n \frac{\sigma_c}{y_c \cos \alpha}  {y_{s_i} \cos \alpha} x_{s_i} A_{s_i}  = 0  \label{eqn:eq_cosalpha} \\
 & \int_{A_c} \frac{\sigma_c}{y_c \cos \alpha} y {\cos \alpha}^2 d A_c + \sum_{i=1}^m n \frac{\sigma_c}{y_c \cos \alpha}  {y_{s_i} \cos \alpha}^2  A_{s_i}   = M \cos \alpha. \nonumber 
\end{align}
 
 e, semplificando i coseni:
 
\begin{align}
 & \int_{A_c} \frac{\sigma_c}{y_c} y  d A_c + \sum_{i=1}^m n \frac{\sigma_c}{y_c }   {y_{s_i} } A_{s_i}  = 0 , \nonumber \\
 & \int_{A_c} \frac{\sigma_c}{y_c} y  x d A_c + \sum_{i=1}^m n \frac{\sigma_c}{y_c } {y_{s_i} } x_{s_i} A_{s_i}  = 0  \label{eqn:eq_semplificate} \\
 & \int_{A_c} \frac{\sigma_c}{y_c} y  d A_c + \sum_{i=1}^m n \frac{\sigma_c}{y_c }   {y_{s_i} }^2  A_{s_i}   = M. \nonumber 
\end{align}
 
Raccogliendo:

\begin{align}
 & \frac{\sigma_c}{y_c} \left[ \int_{A_c}  y  d A_c + \sum_{i=1}^m n {y_{s_i} } A_{s_i} \right] = 0 , \nonumber \\
 & \frac{\sigma_c}{y_c} \left[ \int_{A_c} y  x d A_c + \sum_{i=1}^m n {y_{s_i} } x_{s_i} A_{s_i} \right] = 0  \label{eqn:eq_raccolte} \\
 & \frac{\sigma_c}{y_c} \left[ \int_{A_c} y  d A_c + \sum_{i=1}^m n    {y_{s_i} }^2  A_{s_i} \right]  = M. \nonumber  
\end{align}

A ben guardare, i termini tra parentesi quadre delle \ref{eqn:eq_raccolte} altro non sono che, rispettivamente, i momenti statico rispetto all'asse neutro, d'inerzia centrifugo e d'inerzia rispetto all'asse neutro 
$S_{nn}$, $I_{ns}$ e $I_{nn}$, per cui:

\begin{align}
  \frac{\sigma_c}{y_c} S_{nn} = 0 & \Rightarrow S_{nn} = 0 , \label{eqn:eq_momento_statico} \\
  \frac{\sigma_c}{y_c} I_{ns}  = 0 & \Rightarrow I_{ns}  = 0 \label{eqn:eq_momento_inerzia_centrifugo} \\
  \frac{\sigma_c}{y_c} I_{nn} = M & \Rightarrow \sigma_c = \frac{M y_c}{I_{nn}}. \label{eqn:eq_bernoulli-navier}  
\end{align}

Come conseguenza, $ S_{nn} = 0 $ significa che l'asse neutro � baricentrico della sezione reagente omogeneizzata (e permette di trovare un punto di passaggio dell'asse neutro), $I_{ns} = 0$ implica che l'asse
neutro e l'asse di sollecitazione sono coniugati rispetto all'ellissi centrale d'inerzia (e fornisce $\alpha$), mentre $\frac{\sigma_c}{y_c} I_{nn} = M$ � l'equazione di Bernoulli-Navier per la sezione reagente omogeneizzata e 
permette di ricavare semplicemente $\sigma_c$.
 
Esamineremo a questo punto alcuni casi particolare:
\begin{itemize}
\item sezione rettangolare;
\item sezione a T (solai);
\item sezione a L (travi di bordo).
\end{itemize}

\subsection{Sezione rettangolare}
Per le sezioni rettangolari si distinguono due casi principali, ovvero le travi semplicemente o doppiamente armate. Va sottolineato che le travi semplicemente armate di fatto non esistono, non almeno in senso stretto in quanto, anche per il caso della flessione semplice, un minimo di resistenza a taglio va garantita e quindi serve dell'armatura ``reggistaffa'' anche lungo il lato che teoricamente sarebbe non armato: ci sar� insomma una armatura con un $\phi$ ridotto rispetto alla parte dimensionata a flessione. Un esempio potrebbe essere un $\phi 24$ al lembo teso e un semplice $\phi 6$ al lembo superiore come reggistaffa.

\subsubsection{Sezione semplicemente armata (calcolo elastico)}
In questo caso avremo armatura solo al lembo teso.

Le grandezze principali sono:
\begin{itemize}
\item $h$, altezza totale della sezione della trave;
\item $b$ larghezza della trave;
\item $d$, altezza utile, distanza tra il lembo superiore non armato e il baricentro dell'armatura inferiore;
\item $d''$, copriferro;
\item $x$ distanza dell'asse neutro dal lembo superiore;
\item $A_s$ superficie della sezione dell'armatura.
\end{itemize}

Le sollecitazioni agenti in conseguenza della flessione semplice (momento flettente $M$) lungo l'asse di sollecitazione della trave $s-s$ coincidente con l'asse di simmetria verticale sono $C_c$, compressione agente sulla porzione di CLS compresso al di sopra dell'asse neutro e $T_s$, trazione agente sull'armatura al lembo inferiore. $C_c$ � applicata a $x/3$ dal lembo superiore mentre $T_s$ � applicata nel baricentro dell'armatura.
La distanza tra il punto di applicazione delle due sollecitazioni � detta braccio della coppia interna e indicato con la lettera $z$.

\paragraph{Verifica:} per la verifica della trave abbiamo: 

\begin{description}
\item[Dati] \hfill \\
	$M$; $\sigma_c^R$; $\sigma_s^R$; $b$; $d$; $h$; $d''$; $A_s$.
\item[Incognite] \hfill \\
$(\sigma_c ; \sigma_s) \leq (\sigma_c^R ; \sigma_s)$; $x$.
\end{description}


Per il calcolo di $x$ impongo l'equilibrio alla traslazione:
\begin{align*}
& C_c - T_s = 0 \Rightarrow \frac{1}{2} \sigma_c x b - \sigma_s A_s = 0 \\
\text{con } & \frac{\sigma_c}{x} = \frac{\sigma_s/n}{d-x} \Rightarrow \sigma_s = n \left( d-x \right) \frac{\sigma_c}{x} \\
\text{e quindi } & \frac{1}{2} \sigma_c x b - n \left( d-x \right) \frac{\sigma_c}{x} A_s = 0 \\
\end{align*}

Semplificando le $\sigma_c$ e portando $x$ al denominatore, otteniamo una equazione di secondo grado in $x$ la cui unica radice positiva fornisce la posizione dell'asse neutro.

\begin{align}
& \frac{1}{2} \sigma_c x^2 b - n \left( d-x \right) \sigma_c A_s = 0 \nonumber \\
& \frac{1}{2} x^2 b - n \left( d-x \right) A_s = 0 \nonumber \\
& \frac{1}{2} x^2 b + n A_s x - n d A_s = 0 \nonumber \\
& x = \frac{n A_s}{b} \left[ -1 + \sqrt{1 + \frac{2 b d}{ n A_s }}  \right] \label{val:x_smplflex}
\end{align}

\textbf{La posizione dell'asse neutro dipende in questo caso solo dalle caratteristiche geometriche della sezione della trave e non dalle caratteristiche di sollecitazione!}

Un'altra strada per ottenere la posizione dell'asse neutro � l'utilizzo della  \ref{eqn:eq_momento_statico}, ricordando che i momenti statici e d'inerzia sono relativi alla sezione \textbf{omogeneizzata}:

\begin{equation}
S_{nn} = b x \frac{x}{2} + n \left( d-x \right) A_s = \frac{1}{2} x^2 b + n A_s x - n d A_s = 0
\end{equation}

che di fatto � la stessa equazione dalla quale � stata ricavata la \ref{val:x_smplflex}.

Per trovare $\sigma_c$ basta imporre l'equilibrio alla rotazione rispetto al baricentro dell'armatura inferiore (equilibrio inteso come causa=effetto).

\begin{align}
C_c \left( d - x/3 \right) = M \Rightarrow & \frac{\sigma_c b x}{2} \left( d - x/3 \right) = M \Rightarrow  \nonumber \\
\Rightarrow \sigma_c = \frac{2 M}{bx \left( d - x/3 \right)} & \leq \sigma_c^R \label{val:sigmac_smplflx}
\end{align}

La $\sigma_s$ discende dalla relazione vista prima,
\begin{equation*}
 \sigma_s = n \left( d-x \right) \frac{\sigma_c}{x},
\end{equation*}

e quindi, sostituendo il $sigma_c$ della \ref{val:sigmac_smplflx},

\begin{equation}
 \sigma_s = n  \frac{2 M}{bx^2 \left( d - x/3 \right)} \left( d-x \right) \leq \sigma_s^R. \label{val:sigmas_smplflx}
\end{equation}

Le incognite $\sigma_c$ e $\sigma_s$ potevano essere anche calcolate attraverso l'equazione di \textit{Bernoulli-Navier} (\ref{eqn:eq_bernoulli-navier}), dove $y_c$ non � altro che la distanza dall'asse neutro appena vista ($x$):

\begin{equation}
  \sigma_c = \frac{M}{I_{nn}} x \label{eq:bernoulli-navier_smplflexrect}
\end{equation}

Essendo il momento d'inerzia (trascurando il momento d'inerzia della barra e lasciando solo il relativo momento di trasporto)
\begin{equation}
  I_{nn} = \frac{b x^3}{3} + n A_s (d-x)^2, \label{eq:Inn_smplflex}
\end{equation}

consegue che

\begin{equation}
 \sigma_c = \frac{M}{ \frac{b x^3}{3} + n A_s (d-x)^2 } x \label{eq:sigmac_bernoulli-navier}
\end{equation}

Tale risultato sembra in contrasto con la \ref{val:sigmac_smplflx}. Tuttavia � semplice dimostrare che le due formulazioni sono equivalenti. Infatti, dalla \ref{eqn:eq_momento_statico} risulta che:

\begin{equation}
S_{nn} = 0 \Rightarrow \frac{bx^2}{2} = n A_s \left( d-x \right) \label{eq:Snn}
\end{equation}

Sostituendo la \ref{eq:Snn} nella \ref{eq:Inn_smplflex}, si ottiene:


\begin{align*}
 I_{nn} = & \frac{bx^3}{3} + \frac{bx^2}{2} \left( d-x \right) = \frac{bx^3}{3} + \frac{bx^2}{2} d - \frac{bx^3}{2} x = \\ 
 = & - \frac{bx^3}{6} + \frac{bx^2}{2} d = \frac{bx^2}{2}(1 - \frac{x}{3})
\end{align*}

Sostituendo questo risultato nella \ref{eq:bernoulli-navier_smplflexrect}, ottengo esattamente  la \ref{val:sigmac_smplflx}. 

Da qui discende in calcolo della $\sigma_s$ con la formula vista prima.

\paragraph{Progetto:} \hfill
\begin{description}
\item[Dati] \hfill \\
	$M$; $\sigma_c^R$; $\sigma_s^R$; $b$; $d$; $h$; $d''$; $A_s$.
\item[Incognite] \hfill \\
$(\sigma_c ; \sigma_s) \leq (\sigma_c^R ; \sigma_s)$; $x$.
\end{description}


\begin{appendices}
\chapter{Definizioni ``universali''}
Coefficiente di omogeneizzazione: $n = \frac{E_s}{E_c}$

$\sigma_c^R$

Causa = Effetto

\chapter{Il foglietto di Mao}

Il foglietto di Mao di Nadia Baldassino:

\begin{equation}
0 < \varepsilon_c  \leq \varepsilon_{c2}
\Psi = 1 + \frac{1}{n +1 } \cdot \frac{\varepsilon_{c2}}{\varepsilon_c} \cdot
\end{equation}

\end{appendices}


\end{document}
